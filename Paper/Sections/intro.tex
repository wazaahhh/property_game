\section*{Introduction}
\subsection*{Private Property in Nature and Society}

Private property is a fundamental principle in most human societies. Similarly, in nature, biological systems have developed strategies to enjoy exclusive rights on resources.



Social norms and laws  entitle people to protect their rights to enjoy exclusive possession of goods, freedom to move, to keep their actions and thoughts for themselves, or on the contrary, express themselves, without interference of anyone else. Because the right to private property is so precious, it has to be defended against potential violators. The enforcement of private property can occur at the individual level or through social norms and institutions. And the ability to enforce efficiently private property is often seen as factor for stability in (Western) societies, and as such as a factor of trust in societies. \cite{}. 



The ability to enjoy exclusive consumption of resources is a fundamental principle of nature and society. {\bf [drop examples from nature here]}


 principle in nature and in most human societies. Biological organisms, people and organizations enjoy exclusive rights on resources, such as energy, land, intellectual property, and the right to privacy. Because it confers some exclusive advantage, private property is often challenged, in a number of ways, such as robberies, data breaches, and more generally, privacy violations. Social norms and laws entitle individuals to enforce their exclusive property rights, by themselves, or through institutions.


\subsection*{Enforcement of Private Property}
To ensure trust and cooperation between entities, enforcement occurs

- Immune system $\rightarrow$ make a foreign entity cannot invade a living organism

- laws \& law enforcement 



\subsection*{Private Property vs. Cooperation}
Cooperation doesn't involve sharing, apart sharing ``trust".
 
There is a fundamental tension between cooperation and the violation of private property. Cooperation can only be sustained if an agent can trust other agents, if she knows that other agents cannot take profit from her exclusive right on a private property. 

{\bf How much private property violation can a society society?}

{\bf We formulate the hypothesis that the violation of private property has a negative influence (undermines) on cooperation in nature and society.}

\subsection*{Implementation}
For that, we propose a public good game, in which, agents can steal the ``site" of a another agent, with some probability $s$. 


- importance of saturated versus resourceful environments (maybe to be put further down in the paper) $\rightarrow$ tragedy of the commons !

Below, we model cooperation in a game-theoretical game  way,  and we integrate the influence of the violation of exclusive rights, as well as the level of resources available (to some extent, maybe rephrase as ``opportunities").

This is motivated by the observation that individuals ``best-placed" (with higher pay-off) trigger envy and jealousy by others. $\rightarrow$ actually, there are clusters of cooperators that defectors aim to destroy {\bf intentionally} (versus by chance in previous models).

To improve their situation, individuals are often willing to migrate to a more favorable place. What if this place is already occupied and/or has become favorable precisely because it is occupied (think of a field well labored for years, compared to one, which has been abandoned for a long time) ?

So far, the role of private property has received little attention in game theory.

{\bf [Insert a paragraph on cooperation/tragedy of the commons / private property and how they connect to the model]}

how much ``property enforcement" (by whatever means) is needed ? 

As we will show, cooperation can only be sustained for when high levels of property enforcement exist. We study the influence of property violation in the migration game.
