\section*{Introduction}
In nature and society, biological organisms, people and organizations enjoy exclusive rights on resources, such as energy, land, intellectual property, and the right to privacy (extends to impersonation, etc.). Capacity to monopolize resources.

{\bf more examples in nature}


- Human organizations can achieve a subtle balance of private property and sharing (Governing the commons \cite{ostrom})

- importance of saturated versus resourceful environments (maybe to be put further down in the paper) $\rightarrow$ tragedy of the commons !

Below, we model cooperation in a game-theoretical game  way,  and we integrate the influence of the violation of exclusive rights, as well as the level of resources available (to some extent, maybe rephrase as ``opportunities").

This is motivated by the observation that individuals ``best-placed" (with higher pay-off) trigger envy and jealousy by others. $\rightarrow$ actually, there are clusters of cooperators that defectors aim to destroy {\bf intentionally} (versus by chance in previous models).

To improve their situation, individuals are often willing to migrate to a more favorable place. What if this place is already occupied and/or has become favorable precisely because it is occupied (think of a field well labored for years, compared to one, which has been abandoned for a long time) ?

So far, the role of private property has received little attention in game theory.

{\bf [Insert a paragraph on cooperation/tragedy of the commons / private property and how they connect to the model]}

how much ``property enforcement" (by whatever means) is needed ? 

As we will show, cooperation can only be sustained for when high levels of property enforcement exist. We study the influence of property violation in the migration game.
