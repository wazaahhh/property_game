In many biological and human systems, organisms and individuals find ways to enjoy exclusive {\it private property} rights on the consumption of their own resources. These resources are however subjected to theft. In most human societies, examples of property violations include pickpockets and real-estate thefts, violation of intellectual property, online identity frauds and privacy violation. These private property violations undermine trust and cooperation for the production of collective value. Therefore, besides individual protection against private property violations, most human societies set a level of legal, executive and judicial enforcement, which is considered as sufficient to maintain cooperation. But what should be the ``right" level of private property right enforcement to maintain cooperation among individuals? Or conversely, how much private property violation a cooperative society can afford? And how is this level of enforcement is influenced by the migration range of agents, as well as their density ?\\

\noindent We address this question with a {\it mobility public goods game} in which agents can steal a neighboring occupied site with some probability $s$. We find that cooperation levels exhibits a sharp phase transition, and disappears for some value $s > s*$. The critical value $s*$ varies as a function of the migration distance (positively) and the density of occupied sites (negatively). When we introduce noise (of types $I$ and $II$), which has been found to help the emergence of cooperation \cite{}, we find that $s*$ drops by one order of magnitude. In other words, societies that promote the emergence of cooperation are more sensitive to private property violations.\\


\noindent {\bf to do: }
\begin{itemize}
  \item explain qualitatively the origins of the phase transition, including why games with noise are more sensitive to private property violations.
  \item Since cooperation is very sensitive to even small values of $s$ in case of noise, can we determine a maximum value of $s$ beyond which cooperation does no longer emerge? $\rightarrow$ redo the emergence of cooperation experiment with various values of $s$. (\textcolor{red}{I don't think it makes sense to try this because the result is already somehow there: since there is this drop, we know at about which level of cooperators the phase transition occurs. It's very unlikely that cooperation can grow if it is below this critical level.})
  \item Given that $s*$ varies as a function of density $\rho$, can we introduce a growth function, which involves progressive replication of agents (AB models), and we check the evolution of the right level of enforcement? \textcolor{red}{(next paper)}.
\end{itemize}

