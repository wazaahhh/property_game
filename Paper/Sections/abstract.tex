

In most civilizations, people are recognized the rights to keep private, parts of their lives and belongings. Private property spans from land ownership to intellectual property, to the right to privacy. Usually, private property violations are hardly repressed to ensure sufficiently high levels of trust, which in turns allows cooperative behaviors in societies [...].

Here, we report the fragility (amazing robustness) of cooperation in case of migration involving stealing ``private property", up to a certain threshold. Using a [describe experiment here], we find that this threshold is sensitive to the concentration of empty sites. If there are little empty sites, cooperation is more sensitive to property violations, than when the grid is more sparse.


Our results suggest that ``property violation" destroys cooperation, in particular when the cooperation patterns are more resilient (e.g., with mobility).




[chunk] On the property game, I think the main hypothesis to be tested is whether/how private property is necessary for maintaining trust and cooperation in society? As a corrolloray, what amount of private property violation undermines cooperation? I think the tricky thing here is that one might expect the collaboration/trust requires to give up some private property, but at the same time, it seems that property is necessary to sustain cooperation (It basically makes me think a bit of the work of Ostrom on the commons). We can vary a couple of parameters in addition to probability of expelling. It seems that the density of the grid (number of occupied sites over all sites on the grid) plays a role on how property violations influence cooperation. If this result is confirmed, it's nice because, it tells a nice story on how more isolated agents can "handle" better property violations, by opposition to overcrowded places, in which cooperation tolerates way less property violations. [/chunk]