\section*{Results}
Computer simulations of the ``property game" model show that even low probability $q$ migration to non-empty sites, considerably undermines cooperation: in the {\it empty-cell migration only} configuration ($q=0$), cluster of cooperators form with defectors at the boundaries (Fig. \ref{configurations}A). However when $q>0$, defectors can move inside  clusters of cooperators (Fig. \ref{configurations}B). For sufficiently large $q$, clusters of cooperators are invaded and get destroyed. 

{\bf Should we study $q$ versus the size of clusters? $\rightarrow$ what level of ``robustness" is needed to cope with a given level of $q$? $\rightarrow$ maybe understanding the effects of property violations boils down to uncovering the tradeoff between configurations that generate large clusters, and the intensity of property violation $q$ $\rightarrow$ it naturally extends to a dynamics formulation of the problem: how fast clusters can re-generate (or migrate) as they get invaded by defectors (do clusters actually migrate?)}.

{\bf show results with no imitation and no noise1 (migration only + property violation}

{\bf show results with migration, imitation + no noise1}

{\bf Show results with migration, imitation and noise1}

{\bf Measure distribution of cluster sizes? Migration of clusters vs. evolution of their size (gravity center displacement)? $\rightarrow$ more migration implies more threat therefore more move?}

{\bf [Supplementary Materials?]} Figure \ref{phase_transition} shows the abrupt transition from highly sustained cooperation levels to the successful invasion of defectors as $q$ increases. {\bf [Show also a distribution of the probability of time before cooperators disappear]}

{\bf [Supplementary Materials?] Effects of grid sparsity? $\rightarrow$ more sparse grids prevent larger clusters? or maybe not? surely, more sparse grids reduce the probability of {\it expelling}. }

It is interesting that, while PD games with imitation and noise together promote cooperation (and the outbreak of cooperation) \cite{helbing2009}, they are also much more sensitive to property violations (Fig. \ref{phase_transition}).

{\bf Show difference between property violations and noise2 (probability to move to a randomly chosen site (free or occupied) without considering the expected success).}

{\bf property violations break (clusters of cooperators) $\rightarrow$ is there a chance that breaking clusters (of cooperators) can in fact ``help" the outbreak of cooperation? $\rightarrow$ I guess, it's unlikely. Still, it should be maybe studied}



We want to analyze how cooperation can still emerge in the property game.

\begin{enumerate}
  \item Emergence of cooperation under constraint of property violation ?
  \item properties of phase transitions $\rightarrow$ probability distribution of survival time ?
\end{enumerate}



