\begin{figure}[h]
\begin{center}
%\centerline{\includegraphics[width=12cm]{Figures/CCDF_A.eps}}
\caption{Representative simulation results for the spatial prisoners's dilemma with payoffs $T=1.3$, $R=1$, $P=0.1$, and $S=0$ after $t=200$ iterations. The simulations are for 49x49 grids with 50\% empty sites (see Figure \ref{perc_filled} for other percentages of grid sparsity). At time $t=0$, we assumed that 50\% of the individuals were cooperators and 50\% were defectors. \textcolor{red}{\bf For reasons of comparison, all simulations shown were performed with identical initial conditions and random numbers (red, defector, blue, cooperator; white, empty site; green defector who became a cooperator in the last iteration; yellow, cooperator who turned into a defector $\rightarrow$ So far, the initial conditions are randomly set. Shall I always keep the same initial conditions, at least, for presentation purpose?}.}
\label{configurations}
\end{center}
\end{figure}

\begin{figure}[h]
\begin{center}
%\centerline{\includegraphics[width=12cm]{Figures/CCDF_A.eps}}
\caption{{\bf (a)} Phase transitions at different levels of $s*$ for various level of grid sparsity ($perc_filled = \{0.1,0.3,0.5,0.7,.0.9\}$) for $M=5$. {\bf (b) Phase transitions for various levels of Moore neighborhood $M=\{3,5,7,9\}$}.}
\label{phase_transition}
\end{center}
\end{figure}


\begin{figure}[h]
\begin{center}
%\centerline{\includegraphics[width=12cm]{Figures/CCDF_A.eps}}
\caption{{\bf (a)} Typical time series for $s << s*$,$s<s*$, $s\approx s*$, and $s > s*$ (for $perc_filled=0.5$, $M=5$, $q=0$, and $r=0$). {\bf (b)}  }
\label{phase_transition}
\end{center}
\end{figure}









